%\documentclass{acm_proc_article-sp}
\documentclass{sig-alternate}

\usepackage{url}
\usepackage{float}
\usepackage{caption}
\usepackage{subcaption}
\usepackage{hyperref}

\urlstyle{same}

\hypersetup{
    colorlinks,
    citecolor=black,
    filecolor=black,
    linkcolor=black,
    urlcolor=black
}

\newenvironment{packed_enum}{
\begin{enumerate}
  \setlength{\itemsep}{1pt}
  \setlength{\parskip}{0pt}
  \setlength{\parsep}{0pt}
}{\end{enumerate}}

\newenvironment{packed_item}{
\begin{itemize}
  \setlength{\itemsep}{1pt}
  \setlength{\parskip}{0pt}
  \setlength{\parsep}{0pt}
}{\end{itemize}}

\newenvironment{packed_desc}{
\begin{description}
  \setlength{\itemsep}{1pt}
  \setlength{\parskip}{0pt}
  \setlength{\parsep}{0pt}
}{\end{description}}

% --- Metadata ---
\permission{
  This work is licensed under the Creative Commons Attribution 4.0
  International License. To view a copy of this license, visit
  \url{http://creativecommons.org/licenses/by/4.0/}.
}
\conferenceinfo{CU CS Systems Prelim,}{
  Spring 2014. \\
  \crnotice{Copyright held by author(s).}
}
\copyrightetc{
  University of Colorado, Boulder \\
  \the\acmcopyr
}
\crdata{03/2014}
% --- End of Metadata ---

\begin{document}

\title{System and Data Security - A CS Systems Prelim}

\numberofauthors{1}
\author{
  \alignauthor
  Andy Sayler\\
  \affaddr{University of Colorado}
  \affaddr{Boulder, Colorado}
  \email{andy.sayler@colorado.edu}
}

\maketitle

\begin{abstract}
Security is a core component of modern computer systems. From
protecting our data to securing our communications, security across
the computing spectrum is fundamental to the manner in which we
leverage and trust computers. But the security of modern computing
systems has not come easily: it has been improved slowly over many
years, sometimes at the cost of painful lessons. Furthermore, the
modern state of the art in computing security still leaves much to be
desired. In this paper, I explore the development of the current state
of the art in computer security focusing on four core components:
cryptography, access control, file systems, and usability. Computer
security is an inherently large topic, but these core topics provide a
reasonable basis on which a discussion the modern state of computer
security can be built. In particular, I seek to answer the question:
``How can we secure our systems and data in a robust, comprehensive,
and easy-to-use manner?''. This question in examined from a historical
perspective, as well as the perspective of a modern user with modern
use cases. This paper builds on the background work completed in my
Master's Thesis~\cite{custos-masters}, further extending my analysis
of the current state of the art and hypothesizing on future extensions
to this state.
\end{abstract}

\section{Introduction}
\label{sec:intro}

We have reached the age of ubiquitous computing. There is not a facet
of our lives that is not heavily integrated with the vast computing
networks we have built and continue to expand. From the cell phones we
use to communicate to the web sites on which we mange our life's
savings to the hard drives filled with our photos and personal
documents, computers are not only everywhere, but often at the heart
of our most intimate interactions. As such, the security of our
computing infrastructure is a foremost concern in modern computing
system design. But what is the state of the art in computing security?
And how have we arrived at this state? These are question I address in
this paper through the exploration and analysis of ten significant
publications in the field. In particular, I break my analysis of the
state of the art in computer security into four related topics:
cryptography, access control, file storage, and usability. These
topics provide the underpinnings of the bulk of the modern state of
the art of computer security.

On the topic of cryptography (Section~\ref{sec:crypto}), I present the
basics of modern asymmetric cryptographic systems~\cite{Diffie1976},
extensions to these systems to accommodate the diversification of
trust~\cite{Shamir1979}, and the manner in which these core concepts
can be leverage in access control
applications~\cite{Bethencourt2007}. On the topic of access control
(Section~\ref{sec:ac}), I present the basics of modern access control
models~\cite{Sandhu1996}, the ways in which these models can
incorporate cryptography to avoid the need for a trusted compute
base~\cite{Bethencourt2007}, and the manner in which various access
control schemes have been applied to modern file
systems~\cite{Miltchev2008}. On the topic of file systems
(Section~\ref{sec:fs}), I present an effort to support file system
distribution with minimal trust~\cite{Mazieres1999}, an overview of
the security mechanism employed by a range of modern file
system~\cite{Kher2005}, and the manners in which modern file system
implement access control~\cite{Miltchev2008}. Finally, on the topic of
usability (Section~\ref{sec:mgmt}), I present an early effort to
standardize the basic system security primitives~\cite{Samar1996},
techniques for making security more robust and simpler for the end
user~\cite{Cox2002}, and efforts to unify security primitives across
multiple administrative domains~\cite{Morgan2004}. These ten papers
are by no means the complete body of prior art, but they do elucidate
the core concepts relevant to the question at the core of this
paper. References to other relevant works will be provided where
appropriate, but the bulk of my analyses will focus on the papers
listed above.

In addition to exploring the historic contributions of the work
mentioned above and the current state of the art they represent, I
also suggest possible future expansions of this state. At it's core,
this involves looking at the existing answers to the question question
``How can we secure our systems and data in a robust, comprehensive,
and easy-to-use manner?'', as well as proposing potential new
answers. This hypothesizing is addressed with respect to the four core
topics mentioned above within each of the relevant sections and is
intended to inform potential future research paths and projects.

\section{Cryptography}
\label{sec:crypto}

Cryptography is the basis of much of the modern computer security
landscape. This is largely because it represent a security primitive
that does not rely on trusting specific people, platforms, or systems
in order to securely function. Instead, it requires that we place our
trust in only one thing: the underlying math. This has led to the
proliferation of cryptography as the security primitive on which many
other security features are built.

\subsection{History}

Cryptography has a very long history: there is evidence of societies
employing basic cryptographic systems in order to ``secure'' writing
and messages dating back to thousands of years BCE. These early forays
into cryptography, however, lacked the sound grounding in mathematical
theory that makes cryptography so appealing today. I will thus skip
over the bulk of cryptographic history involving them.

Modern cryptography has its roots in the field of information theory
that begin to develop during WWII and advanced quickly in the post war
years. Much of information theory laid the basis for our ability to
prove that a given cryptographic algorithm requires a certain amount
of effort to crack in the absence of the ``key''. This led to the rise
of mathematically grounded symmetric encryption algorithms, designed
for use with the growing availability of computers, by the early
1970s.

Symmetric cryptography algorithms function on the principle that a
single ``key'' is used to both encrypt and decrypt a message. This key
must be securely stored, or if shared, securely exchanged between
parties. Anyone with the key can decrypt the corresponding ciphertext
the key was used to create. The security of a symmetric encryption
cipher tends to be directly related to the length of the encryption
key: the longer the key, the more secure the data encrypted with it
is.

While symmetric cryptography algorithms are useful in situations where
a single actor will be both encrypting and decrypting a piece of data
(and thus can hold the required key personally), they pose a major
challenge it situations where multiple parties wish to communicate
securely. In this situation, the parties must find a way to securely
communicate the required symmetric key. In the absence of
cryptographic methods, the only way to securely exchange a key while
avoiding both eavesdroppers and interlopers is to meet in person and
exchange the key manually. The tediousness and lack of practicality of
this task, especially in modern digital communication systems where
multiple actors may be continents apart, led researchers to seek a
better method for secure data exchange in the absence of an inherently
secure communication channel.

The major breakthrough in solving this challenge came in 1976 with
Diffe and Hellman's publication of ``New Directions in
Cryptography''~\cite{Diffie1976}. Diffe and Hellman proposed a system
for asymmetric cryptography: a cryptography system in which one key is
used for encryption while a second related key is used for
decryption. When properly designed, it is computationally unfeasible
to derive one of the keys in an asymmetric cryptography system form
the other, allowing a user to publish one of their keys for the public
to consume while keeping their other key private. A member of the
public can then use the user's public key to encrypt a message that
only the holder of the private key will be able to decrypt. If all
members of the public maintain such public/private key pairs, it
becomes possible for any user to send any other user a message that
only the recipient an read without requiring any form of secure
communication channel.

Asymmetric cryptography relies on the existence of ``trapdoor''
functions in order to operate. These functions can be quickly solved
in one direction, but are computationally difficult to reverse without
a special piece of information (e.g. the 'key'). Factoring large
numbers is a classic example of a trapdoor function (and the method on
which many modern public key encryption systems are based). Factoring
large numbers is computationally difficult in cases where some piece
of secret information (e.g. one of the factors) is not known.

Diffie and Hellman proposed a potential implementation of a public key
cryptography system, although the first practical public key crypto
system came a few years latter with the invention of the
RSA~\cite{Rivest1978} algorithm. In addition to public/private key
systems, Diffie and Hellman also proposed a system for joint key
generation where two parties can negotiate a secrete key across an
insecure connection. Like asymmetric cryptography, such a system can
be used to bootstrap secure communications across an insecure
connection by allowing two parties to derive a secret key that can
then be used to facilitate further secret communication using a
symmetric encryption algorithm.

Diffie and Hellman also introduce the concept that asymmetric
encryption can be used to build the two additional core cryptographic
primitives we have come to rely upon: cryptographic verification and
cryptographic authentication. Cryptographic verification (also called
a cryptographic ``signature'') is essentially the reverse of
asymmetric encryption: instead of a member of the public using another
party's public key to encrypt a message that only the target party can
read, the target party uses their private key to encrypt a message
that the public can then decrypt using the target's public key. Since
only the target has access to the private key, and is thus capable of
generating such a message, the target can ``prove'' that a given
message comes from them and that it has not been altered in
transit. Just as asymmetric encryption gives rise to cryptographically
secure signatures, cryptographically secure signatures can give rise
to cryptographically secure authentication. If a user generates a
signed message saying ``I am John'' and sends it to an authentication
server, the server can verify that the message signature is valid by
checking it using John's public key, and thus authenticating John in
the process. The server need only have a list of public keys for each
user. It can then leverage the assertion that only the indented user
has access to the corresponding private key for each of the server's
public keys, and is thus the only one capable of generating a signed
message on the user's behalf, as the basis of user authentication.

Beyond the rise of public key cryptography, one of the other major
cryptographic breakthroughs of the last fifty years was the invention
of cryptographically secure secret sharing schemes. In particular, Adi
Shamir (the 'S' form ``RSA'') proposed a practical and robust secret
sharing scheme in his 1979 paper ``How to share a
secret''~\cite{Shamir1979}. In this work, Shamir lays out the basics
of what has come to be known as Shamir Secret Sharing: a method for
splitting a piece of information up into two or more pieces in a
manner such that holders of any subset of the pieces cannot infer any
information about the pieces they do not hold or the original
information block as a whole. Shamir Secret Sharing allows a user to
divide a piece of D data into N pieces of which K or more pieces can
be used to recompute the original value of D. A user with fewer than K
pieces, however, has no more information about the value of D than a
user with no pieces. This system provides a highly useful method for
distributing information amongst multiple parties or systems in
situations where no single party or system can be fully trusted.

Shamir Secret Sharing, unlike all known asymmetric encryption
techniques, does not rely on computational complexity as the basis of
its security. Instead, it is fundamentally secure based on information
theory principles. Thus, unlike computationally secure systems such as
RSA, Shamir Secret Sharing can not be broken regardless of the amount
of computational power one posses. Shamir Secret Sharing functions on
the basis of defining a polynomial of degree (K-1) over a finite field
with the D data encoded as the first order-zero term. N points are
then selected from this polynomial and distributed to the
participants. Since K points (but no fewer) will uniquely identify the
original polynomial, and thus allow the derivation of D, K users must
combine their pieces in order to re-compute D.

Shamir Secret Sharing (and related systems) are useful in a wide range
of situations where one needs to distribute trust across multiple
entities. In particular, secret sharing techniques are leveraged in
some cryptographically-based access control systems like that
described in~\cite{Goyal2006}. Such systems will be discussed further
in Section~\ref{sec:ac}.

\subsection{State of the Art}

Both symmetric and asymmetric encryption systems have a place in the
modern security landscape: symmetric systems for their performance and
resistance to cryptanalysis and asymmetric systems for their
avoidance of the key exchange problem.  Often symmetric and asymmetric
cryptography are used together, each system playing to its
strength. Symmetric systems are good at quickly and securely
encryption data, making them appropriate for the core of an encryption
system. Symmetric ciphers, however, suffer from a lack of natively
secure method for exchanging the required encryption key. This is
where asymmetric cryptography and related secure key exchange systems
come in handy. These systems provide the basis for securely exchanging
data over insecure channels and they can be thus used to bootstrap
symmetric encryption systems by facilitating the secure exchange of a
symmetric encryption key which can then be used to encrypt the
underlying data. Such systems are common in many modern protocols like
SSL, TLS, PGP, and SSH.

Modern symmetric encryption ciphers like AES (Rijndael), Twofish, or
Camellia are well-established, fast, and secure methods for encrypting
data. Symmetric encryption systems are the preferred means of
encrypting files, hard disks, and other large chunks of data due to
their speed and relative simplicity of implementation. They are also
useful for encrypting streams of data in communication protocols like
TLS/SSL or SSH. They tend to be well understood, and are generally
considered highly secure (at least the well vetted ones). The strength
of a given symmetric cipher is directly related to the length of the
associated encryption key. Common key lengths generally considered
secure today include 128-bit keys, 256-bit keys, 384-bit keys, and
512-bit keys.

Modern asymmetric encryption systems include systems like RSA,
ElGamal, ECDH, ECDSA, or ECIES. These systems are all built atop
various one-way trapdoor functions. RSA is based on
prime-factorization functions, ElGamal is based on discrete logarithm
functions, and ECDH, ECDSA, and ECIES are all based on various
elliptic curve functions. It is often useful to mix cipher suites
relying on different trapdoor functions as a hedge against an
efficient solution to any of the underlying function families being
discovered, thus rendering the encryption using such a family
obsolete. Within a given family the security of an asymmetric cipher
is related to the length of the keys in a key pair: longer keys are
more secure. Standard key lengths for asymmetric keys depend of the
family of functions be used. In prime-factorization based system like
RSA or discreet-log systems like ElGamal, key lengths of 1024-bits,
2048-bits, and 4096-bits are all common (with 2048 considered to be
best practice for general data and 4096 considered to be best practice
for highly sensitive data). Elliptic curve based systems tend to use
shorter keys: recommended sizes range from 160-bits to 512-bits
(similar to symmetric key lengths).

\subsection{Future Extensions}

While both symmetric and asymmetric cryptography are largely settled
art at this point, there are a variety of challenges yet to be solved
in the cryptography research realm. One of the constant threats to the
settled cryptographic art is that of a major breakthrough in solving
the currently ``hard'' problems used as the basis for a variety of
trapdoor functions. One of the proposed methods for solving existing
trapdoor functions more quickly than currently possible is to use
quantum computing approaches. While no practical attacks on existing
cryptographic systems using quantum approaches are known, there are
certainly researchers working on such approaches. There is also work
being done in the opposite direction to try to leverage quantum
techniques to build cryptographic systems that would be effectively
unbreakable due to the underlying quantum limitations of our
universe. Such techniques often involve leveraging the quantum
``observer'' effect to detect eavesdropping on a communication
channel, allowing the users to regenerate keys until they complete the
negotiation unobserved. Again, however, there are no known practical
implementations of quantum cryptography at this time.

One of the other major challenges to cryptography today is the
reliance on good sources of randomness to generate the secure
cryptographic keys used in both symmetric and asymmetric encryption
systems. Finding and harnessing good randomness is challenging since
it depends on having access to an sufficient amount of
entropy. Traditionally, computing systems leverage user interactions
(e.g. mouse movements, key strokes, network interrupts, etc) as the
sources for OS-maintained entropy pools. The rise of cloud computing
systems, however, posses challenges to using such sources: in the
cloud, systems often run in highly homogeneous environments with few
``random'' user interactions. Coming up with secure ways to derive
entropy and provide good randomness is such environments is a topic of
active study. The importance of randomness to modern computing systems
requires us to come up with secure ways of gathering and distributing
entropy. Potential solutions range from building secure
Entropy-as-a-Service systems that gather entropy from sources where it
is abundant and redistribute it to source where it is scarce to using
entropy-generating hardware that relies on unpredictable physical
phenomena (e.g. atomic decay events) to derive randomness.

Finally, while strong cryptographic systems are well understood, there
are a myriad of situations where cryptographic best practices are
ignored or where cryptographic systems are unpractical to use, leading
to security failings. Many of these issues can be linked to usability
challenges within cryptographic systems: if a system is difficult to
use or challenging to deploy, it will often go unused or wind up
misconfigured and insecure. One of the major usability challenges
present in all current cryptography systems is the management of
private/secret keys. Keeping such keys secure while also enabling
their use across a range of modern multi-user, multi-device, ephemeral
resource use cases is extremely challenging and lacks a standard
solution at this time. My previous master's thesis work,
Custos~\cite{custos-masters}, proposes a potential solution to the
secret management problem and remains an area of active
research. Other approaches to alleviating usability issues are
discussed in Section~\ref{sec:mgmt}.

\section{Access Control}
\label{sec:ac}

Over the years, we have developed a range of access control
techniques. All of these techniques share a common goal: controlling
access to a specific system, resource, or piece of data. Must access
control models have two key components: authentication and
authorization.  Authentication is used to establish the identity of an
actor. Authorization then leverages this identification as the basis
of granting or denying specific permissions to the actor. This section
will discuss historic access control techniques, the current state of
the access control art, and potential future access control additions.

\subsection{History}

Computer-based access control systems have been with us since the
earliest multi-user (e.g. time sharing) operating systems became
popular in the 1970s and 1980s. Early access control systems were
primarily focused around the Unix model of access control: users,
groups, and read/write/execute file-level permissions. Authentication
in these early systems was generally limited to username:password
combinations, the mechanisms of which were hard coded into the
\texttt{login} program. Each user was a member of one or more groups
and each file had a owner and group. The three file permissions, read,
write, and execute, were granted on the basis of a user's relationship
to a given file: either the user was the file owner, the user was a
member of the file group, or the user was neither of these
things. This model is fairly flexible, and continues to be used today
as the core access control model in many Unix-like operating systems
(e.g. BSD, Linux, OSX, etc).

Access Control List (ACL) based schemes gained prominence in 1990s and
were popularized by the Windows NT family of operating systems. ACLs
extend the permission model beyond the basic Unix file permissions to
include a wider range of file (e.g. read, write, delete, create, etc)
and system-level (e.g. shutdown, connect to network, etc)
permissions. ACLs are associated with specific system objects
(e.g. files, folders, OS subsystems, etc) and map a user or group to a
list of permission that user or group possess. They generalize the
Unix access model to accommodate a wider range of permissions and
mappings between permissions and actors. ACL-based systems have been
integrated into many modern Unix-like operating systems as an optional
extension beyond the tradition Unix permissioning scheme.

Exiting access control schemes are often grouped into one of two
classes: Mandatory Access Control (MAC) systems or Discretionary
Access Control (DAC) Systems. While the lines between these two
approaches are occasionally blurred, the basic difference between the
two lies in which actors within a system have the ability to
grant/extend permissions to other actors. In MAC system, all
permissions are set by the system administrator and users have no
ability to change these permissions themselves or transfer permissions
to other users. DAC systems, in contrast, give users the ability to
set their own permissions on objects they own or create, and to
transfer these permissions to other users. A MAC-based system can be
thought of as similar to a DAC system where the system administrator
owns all files and never transfer this ownership to any other
user. Traditional Unix access control systems as well ACL access
control systems can generally be used in either MAC or DAC based
systems. MAC systems are generally preferred in high security
environments where the centralized management models they offer lead
to tighter control over data. DAC systems are more common in general
purpose systems where the extra flexibility they offer reduces the
administrative burden. Most Unix-like systems are DAC systems by
design, but extensions (e.g. SELinux) can be used to add MAC
properties to these systems.

Many of the early access control systems pose a host of manageability
challenges. How do you coordinate the permissions of thousands of
users across millions of objects? How do you revoke permissions for a
defunct user? Or add a new user? To cope with many of these challenges
Sandhu, et. al. proposed the concept of Role-Based Access Control
Models in their 1996 paper by the same name~\cite{Sandhu1996}.
Role-Based Access Control (RBAC) inserts an additional layer of
indirection between users and permissions. In an RBAC system, users
are assigned to one or more roles. Each role is then assigned one or
more permissions. This model simplifies management by separating
permission assignment from specific users. RBAC permissions are
assigned assigned on the basis of specific positions or duties within
an organization and mapped to specific roles. Users are then assigned
to these roles on the basis of whether or not they hold a specific
positions or are required to perform a specific duty. Thus, adding or
removing users does not require any modification to permission
mappings, only role mappings. Likewise, adding or removing permissions
does not require modifying user mappings, only role
mappings. \cite{Sandhu1996} describe 4 classes of RBAC systems:
$RBAC_0$ (the base model), $RBAC_1$ (the base model with the addition
of role hierarchies and inheritance), $RBAC_2$ (the base model with
the addition of constraints), and $RBAC_3$ (the combination of
$RBAC_1$ and $RBAC_2$).

The primary limitation of all of the access control models mentioned
thus far is their reliance on a trusted arbiter for enforcement:
generally this trusted arbiter is the operating system or some other
underlying program in charge of enforcing the access control
system. This means that the security of any of these access control
systems is only as good as the security of the system enforcing them
(e.g. the security of the OS). Thus, if the underlying OS or program
is compromised, the access control system falls apart. Likewise,
anyone in control of the underlying OS or program (e.g. an
administrator) automatically gains full control over the access
control system. This is an acceptable limitations in many situations,
especially those based on a centrally managed system with existing
physical security and administrative safeguards in place. But in
distributed systems or other systems where physical and management
control in not guaranteed (e.g. the cloud), a more robust system that
lacks this ``trusted arbiter'' requirement is desirable.

To overcome the need for a trusted enforcement mechanism in access
control systems, researchers have turned to cryptographically-based
access control systems. As mentioned in Section~\ref{sec:crypto},
cryptographic-based system require no trust in external systems, only
in the underlying math. Sahaie, Waters, et. al. propose several
cryptographically-based access control systems in their 2006
paper~\cite{Goyal2006} and its 2007 follow-up~\cite{Bethencourt2007}.
These two systems are based on the concept of Attribute-Based
Encryption (ABE) schemes. ABE schemes allow a user to encrypt a
document in a manner such that the access control rules associated
with the document are part of the encryption process itself. Thus, in
order to decrypt/access a document, a user must satisfy one or more
cryptographically guaranteed access control attributes.
\cite{Goyal2006} allows user to encrypt documents that can only be
decrypted by users possessing specific attribute polices encoded in
their keys. \cite{Bethencourt2007} extended this concept to allow
documents to be encrypted with a full access control policy tree
embedded in the encryption processed file directly allowing only users
who's private keys meet a generalized set of requirements to access
the documents. Both these systems allow the construction of access
control systems that do not require any trusted arbiter to regulate
access to objects. Instead, the regulation in enforced by the
underlying encryption backed by the associated math.

\subsection{State of the Art}

Today, the state of the are in access control differs widely based on
the system and application. As mentioned, many Linux and Unix-like
systems still use basic Unix access control primitives largely
unchanged over the last 20 to 30 years. This is largely due to the
fact that these systems are ``good enough'' for many single-user and
small group environments and the administrative burden of shifting to
a more advanced system is simply not worth the effort. Most Linux
systems today do support extended file ACLs, as well as system-level
ACLs exposed by systems like PolicyKit. These systems allow users to
move beyond the limitations of a pure Unix-like, file-centric access
control scheme. That said, many users never need to deal with these
more advanced systems for the majority of day-to-day use cases.

Windows-based operating systems make extensive use of ACL-based access
control schemes. While, these schemes are useful in the Windows-domain
environments used by most large corporations, many home and leisure
Windows users never have any need to interact with file or system
level ACLs. Modern Windows systems combine RBAC concepts with ACL
concepts by allowing administrators to define role-based ``groups''
that can then be used with specific permission assignments. Most
stand-alone access control systems bundled with specific applications
(e.g. content management systems, etc) also take cues from both RBAC
and ACL access control models. While these system do help to reduce
the management burden of large systems, they are often prone to
misconfiguration, the occurrences of which lead to many of the
security breaches that happen today.

Cryptographically-based access control schemes remain largely an
academic novelty at this point. I am aware of no commercially or
generally deployed software that leverage ABE or similar
cryptographically-based access control schemes as the basis of their
access control models. None the less, the need for robust access
control schemes that can be used across a range of untrusted
infrastructure is only going to increase in our modern cloud-based
computing world. Thus, it is possible that we will see an increase in
the practical deployment of these systems in the near future.

Outside of permission-side access control models, there have also been
advances in the authentication side of access control. Most modern
access control systems support authorization primitives far more
complex than basic username:password combinations. Several such
systems are discussed in more detail in Section~\ref{sec:mgmt}. None
the less, the vast majority of user authorization systems are still
password based. To overcome the well known security deficiencies of
user passwords, multi-factor authentication schemes are staring to
gain prominence on high value target systems (e.g. email accounts,
bank accounts), but such systems are most often optional and are not
in wide use amongst the general population.

\subsection{Future Extensions}

One of the major limitations to most access control system today is
their lack of global name space or rules: access control rules are
currently scoped to the system or administrative domain in which they
are created, with little to no support for wider, globally-enforceable
rules. This creates a host of challenges implementing robust access
control schemes across our modern multi-user, multi-device, loosely
managed environments. As a basic example, traditional Unix file
permissions are useless when used on portable media like USB flash
drives since user IDs, groups, and permissions are all scoped to the
local machine and do not extend to other machines on which you might
wish to access the portable media. Cryptographically-based access
control schemes like ABE are a step toward solving this problem by
removing the need for a trusted host system. Still, such systems still
pose many of the same challenges other cryptographic systems have:
namely how to manage and control access to the underlying private keys
in a secure yet globally accessible manner. In addition, a variety of
system-specific distributed access control schemes will be discussed
in the context of the file systems to which they apply in
Section~\ref{sec:fs}. But few of these systems are generalized enough
for use on a multi-application scale. Providing secure, manageable,
and usable access control systems that can operate on a global scale
across a variety of distributed devices largely remains an open
problem, and finding a solution will continue to increase in
importance as our use cases continue to demand an increasingly global
and distributed perspective.

Beyond globally usable access control schemes, most access control
system today have a fairly clear delineation of authentication and
authorization. While this division makes since from a separation of
duties standpoint and fits well into tradition access control schemes,
it can also limit the expressiveness of an access control system. For
example, strictly separating authentication and authorization makes it
difficult to set up access control rules that depend on more than a
single user's identity (e.g. time dependent, etc) and also make it
difficult to operate in situations where the concept of a single
``user'' is not well defined (e.g. anonymous systems).
In~\cite{custos-masters}, I explored relaxing the separation between
authentication and authorization to build a more expressive access
control scheme. Schemes like ABE also blur the lines between
authentication and authorization in the name of increased
expressiveness. How to strike a correct balance between a proper
separation of authentication and authorization duties and a high level
of expressiveness remains an open problem. Building expressive systems
that are also easy to reason about, manage, and maintain is a relevant
topic to future access control advances.

\section{File Systems}
\label{sec:fs}

Much of the work we perform on computers today is highly
data-centric. As such, the protection and control of our data is a
core goal within the realm of computer security. The previous two
sections explored ways to protect data cryptographically and via
various generalized access control models. In this section, I'll look
at data protection schemes built into storage systems directly.

\subsection{History}

Early storage and file system technologies often simply neglected
security, lacking robust encryption and access control primitives. As
mentioned in Section~\ref{sec:ac}, the rise of multi-user operating
systems like Unix mandated the creation of basic file-system access
control schemes. Thus we gained the previously mentioned Unix file
access control and permissioning scheme as part of the virtual file
system (VFS) abstraction inherent in all legacy and modern Unix-like
operating systems. As previously stated, however, this system has a
number of limitations: it supports only a single, basic access control
model (owner, group, R/W/E permissions), it requires a trusted system
for enforcement, and it is strongly coupled to a local system. Systems
like NFS attempt to extend Unix file security semantics beyond the
local machine allowing remote sharing of files, but even these systems
are limited to singular administrative domains and trusted systems.

The Windows NT file system access control model (implemented via the
NTFS file system) extends the flexibility of the traditional Unix
model by adding support for more expressive ACLs. These both allow the
control of additional permissions (e.g. delete, create directory, etc)
as well as more expressive user to permission mappings beyond the
basic owner/group/other Unix model. Furthermore, the Windows NT model
has the ability to delegate user authentication to a local Domain
Controller (DC) capable of centrally managing all users from a single
location. This expands the ability to control file access beyond the
users associated with the local system to the users associated with an
entire administrative domain. Still, this system still has many of the
same limitations as the Unix model: the requirement for a trusted
system for enforcement and the tight coupling to the local
administrative domain.

The rise of the Internet as a reliable and high speed system for
connecting multiple machines across the world as well as the move
toward cloud computing models where computational resources are
outsourced to dedicated providers has increased the demand for secure
storage systems capable of spanning multiple systems and domains. In
order to overcome the limitations posed by traditional file system
security models and accommodate modern multi-user, multi-system use
cases, researchers have proposed a number of newer systems. These
systems try to address one or more of the limitations mentioned
above. Some of them employ cryptographic security models to overcome
the need for a trusted enforcement system. Others are designed to
extend access control semantics beyond the local machine to large
networks or even the global internet. Still others explore the use of
novel access control models more expressive then Unix permissions or
Windows NT ACLs. Many system combine more than one of these approaches
to build a fully featured next generation secure storage system.

Kher and Kim provide a survey of the various approaches to securing
distributed storage systems in their 2005 paper ``Securing Distributed
Storage: Challenges, Techniques, and Systems''~\cite{Kher2005}. In it,
they discuss the security models of various storage systems, sorting
such systems into basic networked file systems, single-user
cryptographic file systems, and multi-user cryptographic file
systems. As previously mentioned, basic networked file systems rely on
trusted systems and administrators for the enforcement of security
rules. Examples of such systems include the Sun Network File System
(NFS), the Andrew File Systems (AFS), and the Common Internet File
System (CIFS/SMB). All of these systems are designed for use within
local administrative domains and do not scale well to global,
loosely-coupled distributed systems. To deal with the scalability
issues, researchers have built system like SFS (discussed below) or
OceanStore which aim to reduce the administrative burden of large
scale distributed file systems. All of these systems, however, rely on
some degree of system or administrator trust. In order to accommodate
situations where users do not wish to place trust on the underlying
system or remote servers, Kher and Kim discuss a handful of
cryptographically-secure file systems. The best of these systems offer
end-to-end cryptography, meaning that data is encrypted and decrypted
on the client side and the server never has access to the unencrypted
data.  Systems like the Cryptographic File Systems (CFS) provide basic
single-user end-to-end file encryption. While end-to-end encryption is
a powerful security model for enabling secure storage atop untrusted
systems, it does pose challenges with respect to multi-user,
multi-device use cases since it generally requires that all clients
have access to private cryptographic credentials in order to
effectively read or write files. In order to support both end-to-end
encryption and multi-user scenarios, researchers have proposed
multi-user cryptographic storage systems like SiRiUS, Cephus, or
Plutus.

Miltchev, et. al. provide a survey of access control techniques across
a variety of distributed file systems in their 2008 paper
``Decentralized Access Control in Distributed File
Systems''~\cite{Miltchev2008}. They present a framework for analyzing
the suitability of various distributed file systems for modern
multi-user, multi-domain use cases by analyzing five underlying file
system qualities: authentication, authorization, granularity,
delegation, and revocation.  Miltchev, et. al. suggest that any secure
large scale file system must successfully address the functionality of
all five of these factors across multiple administrative domains in
order to be an effective multi-user, multi-domain file system. In
addition to the systems discussed in~\cite{Kher2005}, Miltchev,
et. al. also discuss systems like Truffles, Bayou, WebFS, CapaFS,
DisCFS, WebDAVA, and Fileteller as examples of systems that attempt to
support multi-domain, multi-user, globally-accessible use
cases. Miltchev, et. al. reach the following conclusions regrading
successful secure multi-user, multi-domain file systems: the use of
public-key cryptography for user authentication is an effective way to
support autonomous delegation, capability-based access control systems
tend to lack support for auditing and accountability, ACL-based access
control systems pose scalability challenges when used across
administrative domains, and revocation and user autonomy are often at
odds.

A good example of a modern multi-user distributed file system with
many of the desirable qualities discussed previously is SFS, described
by Mazi\`{e}res, et. al. in their 1999 paper ``Separating Key
Management form File System Security''~\cite{Mazieres1999}. In this
paper, they describe the design and implementation of the SFS file
system. SFS is unique in that it is designed for global-scale file
system operations and leverages basic cryptographic security while
avoiding the need for tightly specified key management
infrastructure. Such infrastructure often adds a large administrative
burden, limiting file system expansion beyond a single administrative
domain on the basis of management complexity alone. SFS achieves its
goals through the use of self-certifying file names: file names that
encode the public key of remote a file server into the file path
itself. Self-certifying path names allow SFS clients to bootstrap a
cryptographically secure communication channel to any remote server
without requiring any large scale key-management infrastructure. SFS
leverages authentication servers to verify users on the basis of
asymmetric key pairs and uses security agents to help reduce the
usability burden this might otherwise impose (see
Section~\ref{sec:mgmt} for more details on agents). SFS is built atop
the NFS distributed file system, and thus uses an a Unix-based access
control model to regulate file and directory access. SFS's
cryptographic capabilities help it to overcome the need for a fully
trusted compute base for the enforcement of its security
model. Extensions to SFS like GSFS help further move SFS beyond the
limitations of a single administrative domain to accommodate a more
general global file system. SFS does not however provide full
end-to-end encryption, so it still requires some trust of the
underlying infrastructure and administration.

\subsection{State of the Art}

As mentioned in Section~\ref{sec:ac}, the basic Unix and Windows NT
file system security models are by far the most widely deployed file
system security approaches. Where distributed file systems are
concerned, NFS, AFS, and CIFS are the de facto standards. As
discussed, all of these systems lack support for end-to-end
cryptographic security, global access, and inter-domain sharing. They
are the standard not because of their rich, modern feature sets (which
they lack), but simply because they are adequate for many of the
standard single user or single domain use cases still deployed
today. Replacing these systems would often be a larger burden than it
is worth. These are not ideal systems, but they are ``good enough''
for many users, and thus remain the standard.

In many instances, the traditional systems listed above are not so
much threatened by the more advanced research systems previously
discussed (e.g. SFS, Bayou, Plutus, etc) but by the growing
proliferation of cloud-based distributed storage service like Dropbox
or Google Drive. Today, when users wishes to share files across
administrative boundaries (or even within these boundaries) or wishes
to sync files across multiple devises, they often do so using a
dedicated file sharing/syncing service based in the cloud. These
services tend to have highly polished user interfaces and overcome
many of the single-domain, single-system restrictions of more
traditional approaches like NFS or CIFS. While polished and easy to
use, these systems do require placing full trust in the cloud
providers who operate them. While most users seem willing to trade
trust and privacy for the features these service provide, there is a
growing understanding (a la Edward Snowden and our friends at the
National Security Agency) of the privacy risk offloading data to the
cloud involves. While these risks don't seem to deter a majority of
users (at least this time), they do make such service strictly off
limits within certain high security and regulatory realms. To date,
none of the major cloud storage services offer access to end-to-end
encryption schemes that would allow users to store their data in the
cloud while also minimizing their need to trust a given cloud service
provider. The reasons behind the lack of a cryptographically secure
cloud service seems to be a combination of lack of user demand,
usability challenges, and the conflicts of interested between
privacy-minded user and data-mining based cloud business models.

While cryptographically secure distributed or cloud-based file systems
are not in wide spread use, there are a growing number of
cryptographically secure local file systems worth mentioning. On
Linux, systems like eCryptfs provide an overlay file system capable of
performing transparent encryption on a subset of the system's file
tree. Alternatively, systems like dm-crypt provide block-level
encryption suitable for full-disk encryption schemes. Using a local
disk encryption system helps to guard against data loss or compromise
in the event that a storage device falls into an advisories
hands. Such systems are particularly useful on mobile computing
devices like laptops or tablets since these devices tend to be more
prone to theft or loss. Systems like BitLocker on Windows or FileVault
on OSX can be used to provide similar features. Even most modern
SSD-based hard disks tend to offer on-board encryption features to
help protect them in the event that they are lost. All of these
systems tend to be fairly user friendly and/or transparent. They often
tie encryption keys to a user's login password, allowing a user to
encrypt their system without requiring any more effort than they would
exert during a normal password-based login process. This approach,
however, does mean that such systems are not useful for protecting
files that can be stolen while a user is actively logged into a system
(e.g. via a malicious program). They only provide protection when a
system is powered off, locked, or similarly put into a non-active
state.

\subsection{Future Extensions}

There is a lot of work to be done in order to make cryptographically
secure, distributed file systems a day to day reality for most
users. While theoretical research systems exist that can provide many
benefits over the current status quo, many of these systems introduce
usability challenges that make them unsuitable for the average
user. As mentioned in previous sections, many of these usability
challenges can be directly linked to the problems that arise managing
private cryptographic keys across our modern multi-device, multi-user
use cases. Usability is the major hurdle preventing cryptographically
strong distributed storage systems form entering the
mainstream. Providing systems that can offer cryptographic security
while also achieving the kind of intuitive usability provided by
service like Dropbox or Google Drive needs to be a major area of
research if we wish for such systems to provide a benefit other than
supporting tenure-track progress within the ivory tower.

The rise of cloud-based services is only going to increase the demand
for secure storage systems that can support multi-user, multi-device,
multi-domain user cases while also minimizing the need to trust cloud
providers. Providing such a system will open up the growing range of
cloud service to an even wider audience, and will help to keep users
in control of their data even as we increasingly outsource our
computing resources. Custos~\cite{custos-masters} was one attempt at
building a component of such a system. Other attempts and further
research are required if we're serious about closing the cloud vs
trust divide.

\section{Usability}
\label{sec:mgmt}

There's little use in having highly secure systems that are impossible
to manage/use. Unfortunately, manageability and ease of use is a major
challenge on many secure systems in use today. Multiple research
efforts have shown that one of the most common causes of security
failures is misconfiguration. Likewise, there have been multiple
studies showing the unwillingness of users to adopt new security
practices if such practices increase the usage burden of a system. In
this section, I'll discuss how secure systems can be made more usable
across developer, administrative, and end-user use cases.

\subsection{History}

The usability of security has been a concern since the early days of
thinking about computer security. Unfortunately, it hasn't always been
a priority. There are multiple facets of usability with respect to
security. In particular, it is useful to analyze the usability of a
system from the frames of reference of various stake holders.  There
are three core stake holders who are affected by the usability of a
security scheme: developers, administrators, and end-users.
Developers care about the programmatic usability of a system: e.g. how
easily can a security scheme be integrated with other systems?
Administrators care about the management usability of a system:
e.g. How easily can a security scheme be installed, configured, and
managed? Finally end-users care about the consumer usability of a
system: which is really to say, they'd generally prefer not to have to
care about a secure system at all. In all these cases, usability is of
the utmost importance. Security is generally a secondary usage goal:
it's not the primary feature a system is trying to implement. Instead,
it's a means to an end. Thus, if security systems become too
burdensome, they will simply be ignored or removed. Security that gets
in the way of the user instead of naturally accommodating the user is
security that will go unused.

As mentioned in previous sections, early computer security
implementations often involved hard coding security primitives
directly into the program that used them. Unfortunately, this practice
quickly leads to developer usability challenges, tightly coupling
security policy with security mechanism and making it difficult to
update one without changing the other. To deal with issues like this,
Vipin Samar invented the Pluggable Authentication Modules (PAM) system
and presented it in his 1996 paper~\cite{Samar1996}. The PAM system
was designed to provide a flexible interface for user authentication
on Unix (and later Linux) systems. It strips the \texttt{login}
program of its internal authentication mechanism code, instead
providing it with an interface into an extensible authentication
plugin stack. Like the GSS-API system that compliments it, PAM is
designed to provide a generic security mechanism (in this case, a
generic authentication mechanism) so that individual programs do not
have to hard-code their authentication algorithms. In this manner, PAM
systems ease developer usability by freeing programmers from the
burden of having to code an authentication stack directly into their
applications. This also avoids the need to constantly upgrade
applications simply to support new authentication
mechanisms. Furthermore, PAM expands the administrator's ability to
control how authentication mechanisms are used within each
application. It allows administrators to control exactly which
authentication primitives get used by each application on a system,
even allowing them to specific multiple authentication primitives
where required (e.g. to provide multi factor authentication). The
manner in which PAM does this is fairly straightforward, ideally
promoting administrative usability in addition to developer usability.

Beyond the developer and administrative usability of authentication
systems lies a whole suite of usability challenges when it comes to
having users managing passwords and authentication credentials
(e.g. keys) for a range of services. Users are often asked to remember
a wide range of passwords for a variety of services. When using
cryptographically based authentication schemes (e.g. public/private
keys/certificates), users must also keep track of all the necessary
private keys so that they can use them when required. Thus, the rather
mundane task of password or certificate based authentication leads to
a large end-user burden, encouraging the creation of simple,
repetitive passwords or discouraging the use of cryptographic
certificates. Cox, et. al present a potential solution to these kinds
of problems in the 2002 paper on ``Security in Plan
9''~\cite{Cox2002}. Cox, et. al. propose using a security ``agent'' to
manage the complexity of maintaining multiple authentication secrets
(e.g. passwords or keys) on behalf of a user. They build such an agent
for the Plan 9 operating system called Factotum. The idea of having a
dedicated ``agent'' manage credentials on the user's behalf predates
Factotum (i.e. the SSH agent), but Factotum generalizes the concept to
make it an application-agnostic credential manager. A security agent
like Factotum locally stores and tracks all credentials for a given
user. These credentials are encrypted, generally using the user's
login password, ensuring that the agent has access to them when the
user is logged in, but that they remain secure when the user is
offline. When the user attempts to authenticate to a service that
requires credentials, Factotum looks them up and provides them on the
user's behalf. If Factotum lacks the necessary credentials, it prompts
the user for them, and then stores them for future use. In this
manner, Factotum can make authentication to a range of disparate
services a largely seamless task for the user, greatly easing end-user
usability and encouraging the use of strong password or cryptographic
authentication techniques that would otherwise be too burdensome to
leverage.

Morgan, et. al. present an alternate approach to managing the
authentication across a range of service in their 2004 paper
``Federated Security: The Shibboleth Approach''~\cite{Morgan2004}.
Where as agent-based system like Factotum deal with the multitude of
user credentials on the client side, Shibboleth attempts to deal with
this problem from the server side. Shibboleth is a from of advanced
Single-Sign-On (SSO) system that allows users to use a single set of
credentials to authenticate to a range of Shibboleth-backed sites. In
contrast to more traditionally SSO systems, Shibboleth offers support
for federations: the ability to share identify information across
administrative domains. In Shibboleth, each users is associated with
an Identity provider (IdP) server. This server maintains identity and
authentication information about a user. When a user wishes to
authenticate to a third party website, that website looks up and
contacts the user's IdP server. The user's session is passed to the
IdP server where user completes the authentication process, after
which they are passed back to the original website with a token
(i.e. assertion) stating whether or not their authentication attempt
was successful and what access attributes they should be granted. The
original site then grants access based on the validity of this token
and the associated attributes it provides. Shibboleth leverages the
Security Assertion Markup Language (SAML) on the back-end as a
standardize format for passing security assertions between an IdP and
a third party site. Since Shibboleth only requires that a user
maintain and provide a single set of authentication credentials
regardless of the number of disparate sites or services they need to
access, it can greatly reduce the user's security burden and encourage
the use of more secure (albeit harder to remember) credentials. It
also eases developer usability by relieving services of the need to
build their own authentication systems (similar to PAM) and
administrative usability by providing a centralized point at which all
user attributes and permissions can be managed.

\subsection{State of the Art}

Today, a wide range of security management systems designed to
increase various usability aspects are in common use. PAM is still the
de facto standard for managing authentication on Unix and Linux
systems. Today, PAM supports a wide range of authentication
primitives, from passwords, to multi-factor devices, to hardware-based
SmartCards. PAM support authentication for a range of Unix and Linux
subsystems including \texttt{login}, SSH, FTP, etc. Windows provides
similar pluggable authentication interfaces. All of these systems
encourage the rapid development of authentication primitives without
unnecessarily tying them to the associated program leveraging
them. They also expose a lot of administrative flexibility when it
comes to configuring authentication across a range of services.

While the concept of an OS-managed general security agent such as
Factotum hasn't really gone mainstream, specific systems make use of
agents for credential management, namely SSH and GnuPG. Password
mangers, however, have become increasingly mainstream. These system
specialize in the storage of passwords and other secrets, and while
they often lack the fully integrated nature of an authentication
agent, they do tend to support rudimentary auto-completion of user
secrets where required. Since the bulk of day-today user
authentication today takes place on the web, most password managers
are browser based. In many ways, the browser in the modern ``OS'' for
many user activities. In that manner, browser-based password managers
such as LastPass or 1Password fulfill many of the same goals as
Factotum: they ease the end-user's security burden and encourage
better security practices in the process. Most security experts
recommend the use of a password manager for all users today. The fact
that many users use the same password across multiple sites and choose
essentially weak passwords is a major barrier to internet
security. Password managers help counter this weakness by focusing all
of the user's memorization efforts in a single, strong master password
while encourage the use of long, random passwords between the password
management software and the third part web site or service.

The Shibboleth system is deployed and in use across a range of
academic and Internet2 infrastructure. Similar federated identity
protocols such as OpenID or Persona have become common across the
wider internet. Large cloud providers like Google or Facebook often
act as identify providers for users who already have accounts with
them, allowing these users to authenticate to other websites without
creating additional accounts. Identity provision is quickly becoming a
core role of large cloud service providers, and options like ``Login
with Facebook'' or ``Login with Google'' are fairly ubiquitous across
web services. These systems benefit users by reducing the number of
credentials they must remember. When coupled with password managers
and/or multi-factor authentication systems, they provide the basis for
a much stronger web authentication framework than forcing users to
remember a different password for each web site they utilize would
allow.

\subsection{Future Extensions}

Usability across administrative, developer, and end user domains
remains a challenge in the security realm. Building secure systems
that are easy to use, easy to manage, and easy to build is a
prerequisite to having any security system gain wide spread
adoption. Many otherwise suitable security approaches have been doomed
by the fact that they posed serious usability issues for one or more
of the aforementioned groups.

One of the major remaining usability challenges is deciding on the
best manner in which to manage user secrets and credentials across the
multitude of devices today's user expect to use. Traditional agent
programs generally fail in multi-device use cases since these system
keep only a local cache of user secretes that is useless if the user
is trying to authenticate from a system other than the one on which
they originally provided their credentials. This calls for a form of
agent program capable of syncing data across multiple devices, while
also retaining the security of the data should one of the devices
become compromised.

Systems like browser-based password managers or federated identity
services overcome the multi-device issue by providing a central, often
cloud based, repository of user credentials or identity data. But this
introduces a new issue: trust. Do we really want to be trusting a
single third party cloud provider with all of our credentials, or with
our identity and it's accompanying metadata? While convenient, such
trust does seed user privacy and control, and potentially increases
the risk of user data exposure should a third party lack scruples,
fall to an attack, or be legally compelled to provide user data.

The best manner in which to provide the kind of convenience security
agents, password managers, and identity providers offer while also
supporting multi-device use cases and minimizing third part trust
requirements remains an open question. One potential approach to
building such a system would be to leverage one of the aforementioned
cryptographically secure storage technologies as the basis of
multi-device secret manager. Doing this in a manner that does not
increase user burden (by forcing them to operate their own storage
infrastructure) will not be easy. Peer-to-peer systems like BitTorrent
Sync might provide avenues toward building such a system, but such a
system has yet to be seriously proposed or prototyped. Furthermore,
most cryptographically secure storage systems require some form of
private key management, which in and of itself can pose usability
challenges. This creates a bit of a ``chicken and the egg'' problem
where technologies to provide security in the absence of third party
trust require just such systems to reduce the end-user usage
burden. The solution to this cyclical usability vs trust problem is a
potential area of active research (possible avenues for with are
discussed in~\cite{custos-masters}).

\section{Conclusion}
\label{sec:conclusion}

Computer security has been said to be like insurance: no one wants to
deal with it until they need it, and by then it's too late. Ensuring
that we can build secure computing systems is going to be the
cornerstone or computing's success in the future. If computers and the
service we build atop them can not be trusted to remain secure, they
will be abandoned as tools of serious work. In order to advance the
state of the art in computer security, I propose focusing on the
following problems. These topics cut across the previously discussed
security realms and help to elucidate the originally posed question,
``How can we secure our systems and data in a robust, comprehensive,
and easy-to-use manner?''

\begin{packed_desc}
\item{Multi-User|Multi-Device|Multi-Domain}: \hfill \\ Most dominant
  use cases today require support for all of the ``multi-*'' scenarios
  listed above. Users expect the ability to share and collaborate with
  other users, users expect system to work across a range of personal
  computing devices, and users do not wish to be burdened by arbitrary
  delineations like a specific administrative domain. Any successful
  security design must accommodate all of these desires gracefully if
  it is to succeed.
\item{Control of Trust}: \hfill \\ The rise of the cloud has provided
  numerous benefits and conveniences to most end users. Unfortunately,
  it often does so at the expense of allowing users to decide whom to
  trust and to place in control of their data. Successful security
  systems must grant users the leeway to pick and chose whom to trust
  and how their data can be used.
\item{Usability}: \hfill \\ Any security system that significantly (or
  for that matter, even moderately) increases the effort a user must
  expend to complete a task is doomed to fail. End-users, developers,
  and administrators do not generally enjoy having to think about
  security. Successful security systems must be easy to use and avoid
  subjecting users to onerous burdens if they are to be adopted.
\end{packed_desc}

These goals are lofty, but I believe they are attainable. Future
research work focused around accomplishing one or more of them will
provide the basis of the future of computing security, and by
extension, the future of computing in general. Working toward the
betterment of computer security is an important pursuit and one that
must be undertaken to ensure the long term viability of modern
computing practices.

\vfill
\break

% Bibliography
\bibliographystyle{acm}
\bibliography{prelim}

\end{document}

%%  LocalWords: BCE Diffe Hellman's Diffie Custos AFS CryptoCache VFS
%%  LocalWords: OceanStore SFS Plutus Adi Shamir ElGamal ECDH ECDSA
%%  LocalWords: ECIES RBAC ACL DAC SELinux Sandhu Coyne Feinstein
%%  LocalWords: Youman Sahaie PolicyKit NTFS Kher CIFS SMB CFS WebFS
%%  LocalWords: SiRiUS Cephus Miltchev CapaFS DisCFS WebDAVA Mazi dm
%%  LocalWords: Fileteller SFS's GSFS Snowden BitLocker FileVault
%%  LocalWords: SSD Vipin GSS SSO IdP LastPass
