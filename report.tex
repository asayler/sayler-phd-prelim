%\documentclass{acm_proc_article-sp}
\documentclass{sig-alternate}

\usepackage{url}
\usepackage{float}
\usepackage{caption}
\usepackage{subcaption}
\usepackage{hyperref}

\urlstyle{same}

\hypersetup{
    colorlinks,
    citecolor=black,
    filecolor=black,
    linkcolor=black,
    urlcolor=black
}

\newenvironment{packed_enum}{
\begin{enumerate}
  \setlength{\itemsep}{1pt}
  \setlength{\parskip}{0pt}
  \setlength{\parsep}{0pt}
}{\end{enumerate}}

\newenvironment{packed_item}{
\begin{itemize}
  \setlength{\itemsep}{1pt}
  \setlength{\parskip}{0pt}
  \setlength{\parsep}{0pt}
}{\end{itemize}}

\newenvironment{packed_desc}{
\begin{description}
  \setlength{\itemsep}{1pt}
  \setlength{\parskip}{0pt}
  \setlength{\parsep}{0pt}
}{\end{description}}

% --- Metadata ---
\permission{
  This work is licensed under the Creative Commons Attribution 4.0
  International License. To view a copy of this license, visit
  \url{http://creativecommons.org/licenses/by/4.0/}.
}
\conferenceinfo{CU CS Systems Prelim,}{
  Spring 2014. \\
  \crnotice{Copyright held by author(s).}
}
\copyrightetc{
  University of Colorado, Boulder \\
  \the\acmcopyr
}
\crdata{03/2014}
% --- End of Metadata ---

\begin{document}

\title{System and Data Security - A CS Systems Prelim}

\numberofauthors{1}
\author{
  \alignauthor
  Andy Sayler\\
  \affaddr{University of Colorado}
  \affaddr{Boulder, Colorado}
  \email{andy.sayler@colorado.edu}
}

\maketitle

\begin{abstract}
Security is a core component of modern computer systems. From
protecting our data to securing our communications, security across
the computing spectrum is fundamental to the manner in which we
leverage and trust computers. But the security of modern computing
systems has not come easily: it has been learned and improved slowly
over many years, sometimes at the cost of painful
lessons. Furthermore, the modern state of the art in computing
security still lea vs much to be desired. In this paper, I explore the
development of the current state of the art in computer security
focusing on four core components: cryptography, access control, file
system security, and security management. Computer security is an
inherently large topic, but these core topics provide a reasonable
basis for the modern state of computer security. In particular, I seek
to answer the question: ``How can we secure our systems and data in a
robust, comprehensive, and easy-to-use manner?''. This question in
examined from a historical perspective as well as the perspective of a
modern user with modern use cases. This paper builds on the background
work completed in my Master's Thesis~\cite{custos-masters}, further
extending my analysis of the current state of the art and
hypothesizing on future extensions to this state.
\end{abstract}

\section{Introduction}
\label{sec:intro}

We have reached the age of ubiquitous computing. There is not a facet
of our lives that is not heavily integrated with the vast computing
networks we have built and continues to expand. From our cell phones
we use to communicate to the web sites on which we mange our life's
savings to the hard drives in our laptops filled with our photos and
personal documents, computers are not only everywhere, but often at
the heart of our most intimate interactions. As such, the security of
our computing infrastructure is a foremost concern in modern computing
system design. But what is the state of the art in computing security?
And how have we arrived at this state? These are question I address in
this paper through the exploration and analysis of ten significant
publications in the field. In particular, I break my analysis of the
state of the art in computer security into four related topics:
cryptography, access control, storage security, and managing
security. These topics provide the basis of the bulk of the modern
state of the art of computer security.

On the topic of cryptography (Section~\ref{sec:crypto}), I present the
basics or modern cryptographic systems~\cite{Diffie1976}, extensions
to these systems to accommodate the diversification of
trust~\cite{Shamir1979}, and the manner in which these core concepts
can be leverage in access control
applications~\cite{Bethencourt2007}. On the topic of access control
(Section~\ref{sec:ac}), I present the basics of modern access control
models~\cite{Sandhu1996}, the ways in which these models can
incorporate cryptography to avoid the need for a trusted compute
base~\cite{Bethencourt2007}, and the manner in which various access
control schemes have been applied to modern file
systems~\cite{Miltchev2008}. On the topic of storage security
(Section~\ref{sec:fs}), I present an effort to support file system
distribution with minimal trust~\cite{Mazieres1999}, an overview of
the security mechanism employed by a range of modern file
system~\cite{Kher2005}, and the manners in which modern file system
implement access control~\cite{Miltchev2008}. Finally, on the topic of
management (Section~\ref{sec:mgmt}), I present an early effort to
standardize the basic system security primitives~\cite{Samar1996},
techniques for making security more robust and simpler for the end
user to leverage~\cite{Cox2002}, and modern efforts to unify security
primitives across multiple administrative
domains~\cite{Morgan2004}. These ten papers are by no means the
complete body of prior art, but they do elucidate the core concepts
relevant to the question at the core of this exam. References to other
relevant works will be provided where appropriate, but the bulk of my
analyses will focus on the papers listed above.

In addition to exploring the historic contributions of the work
mentioned above and the current state of the art they represent, I
also suggest possible future expansions of this state. At it's core,
this involves looking at the existing answers to the question
question: ``How can we secure our systems and data in a robust,
comprehensive, and easy-to-use manner?'' as well as proposing
potential new answers. This hypothesizing is addressed with respect to
the four core topics mentioned above within each of the relevant
sections. This hypothesizing is intended to inform potential future
research paths and projects.

\newpage

\section{Cryptography}
\label{sec:crypto}

Cryptography is the basis of much of the modern computer security
landscape. This is largely because it represent a security primitive
that does not rely on trusting specific people, platforms, or systems
in order to securely function. Instead, it requires that we place our
trust in only one thing: the underlying math. This has led to the
proliferation of cryptography as the security primitive on which many
other security features are built.

\subsection{History}

Cryptography has a very long history: there is evidence of societies
employing basic cryptographic systems in order to ``secure'' writing
and messages dating back to thousands of years BCE. These early forays
into cryptography, however, lacked the sound grounding in mathematical
theory that makes cryptography so appealing today. I will thus skip
over the bulk of cryptographic history involving them.

Modern cryptography has its roots in the field of information theory
that begin to develop during WWII and advanced quickly in the post war
years. Much of information theory laid the basis for our ability to
prove that a given cryptographic algorithm requires a certain amount
of effort to crack in the absence of the ``key''. This led to the rise
of mathematically ground symmetric encryption algorithms, designed for
use with the growing availability of computers, by the early 1970s.

Symmetric cryptography algorithms function on the principle that a
single ``key'' is used to both encrypt and decrypt a message. This key
must be securely stored, or if shared, securely exchanged between
parties. Anyone with the key can decrypt the corresponding ciphertext
the key was used to create. The security of a symmetric encryption
cipher tends to be directly related to the length of the encryption
key: the longer the key, the more secure the data encrypted with it
is.

While symmetric cryptography algorithms are useful in situations where
a single actor will be both encrypting and decrypting a piece of data
(and thus can hold the required key personally), they pose a major
challenge it situations where multiple parties with to communicate
securely. In this situation, the parties must find a way to securely
communicate the required symmetric key. In the absence of other
methods, the only real way to do while avoiding both eavesdroppers and
interlopers is to meet in person and exchange the key manually. The
tediousness and lack of practicality of this task, especially in
modern digital communication systems where multiple actors may be
continents apart, led researchers to seek a better method for secure
data exchange in the absence of an inherently secure communication
channel.

The major breakthrough in solving this challenge came in 1976 with
Diffe and Hellman's publication of ``New Directions in
Cryptography''~\cite{Diffie1976}. Diffe and Hellman proposed a system
for asymmetric cryptography: a cryptography system in which one key is
used for encryption while a second, related, key is used for
decryption. When properly designed, it is computationally unfeasible
to derive one of the keys in an asymmetric cryptography system for the
other, allowing a user to publish one of their keys for the public to
consume while keeping their other key private. A member of the public
can then use the user's public key to encrypt a message that only the
holder of the private key will be able to decrypt. If all members of
the public maintain such public/private key pairs, it become possible
for any user to send any other user a message that only the recipient
an read without requiring any form of pre-secure communication
channel.

Asymmetric cryptography relies on existence of ``trapdoor'' functions
in order to operate. These functions can be quickly solved in one
direction, but are computationally complex to reverse without a
special piece of information (e.g. the 'key'). Factoring large numbers
is a classic example of a trapdoor function (and the method on which
many modern public key encryption systems are based). Factoring large
numbers is computationally difficult in cases where some piece of
secret information (e.g. one of the factors) is not known.

Diffie and Hellman proposed a potential implementation of a public key
cryptography system, although the first practical public key crypto
system came a few years latter with the invention of the
RSA~\cite{Rivest1978} algorithm. In addition to public/private key
systems, Diffie and Hellman also proposed a system for joint key
generation where two parties can negotiate a secrete key across an
insecure connection. Like asymmetric cryptography, such a system can
also be used to bootstrap secure communications across an insecure
connection by allowing two parties to derive a secret key that can
then be used to facilitate further secret communication using a
symmetric encryption algorithm.

Diffie and Hellman also introduce the concept that asymmetric
encryption can be used to build the other key cryptographic primitives
we have come to rely upon: cryptographic verification and
cryptographic authentication. Cryptographic verification (also called
a cryptographic ``signature'') is essentially the reverse of
asymmetric encryption: instead of a member of the public using another
party's public key to encrypt a message that only the target party can
read, the target party uses their private key to encrypt a message
that the public can then decrypt using the target's public key. Since
only the target has access to the private key, and is thus capable of
generating such a message, the target can ``prove'' that a given
message comes from them and that it has not been altered in
transit. Just an asymmetric encryption also gives us cryptographically
secure signatures, cryptographically secure signatures can give us a
form of cryptographically secure authentication. If a user generates a
signed message saying ``I am John'' and sends it to an authentication
server, the server can verify that the message signature is valid by
checking it with John's public key, thus authenticating John. The
server need only have a list of public keys for each user and can then
use the fact that only the given user has access to the corresponding
private key and is thus the only one capable of generating a signed
message on the user's behalf as the basis of user authentication.

Beyond the rise of public key cryptography, one of the other major
cryptographic breakthroughs of the last fifty years was the invention
is cryptographically secure secret sharing schemes. In particular, Adi
Shamir (the 'S' for RSA~\cite{Rivest1978}) proposed a practical and
robust secret sharing scheme in his 1979 paper ``How to share a
secret''~\cite{Shamir1979}. In this work, he lays out the basics of
what came to be known as Shamir Secret Sharing: a method for splitting
a piece of information up into two or more pieces in a manner such
that holders of any subset of the pieces cannot infer any information
about the pieces they do not hold or the original information block as
a whole. Shamir Secret Sharing allows a user to divide a piece of data
D into N pieces of which K or more pieces can be used to recompute the
original value of D. A user with fewer than K pieces, however, has no
more information about the value of D than a user with no pieces. This
system provides a highly useful method for distributing information
amongst multiple parties or systems in situations where no single
system can be fully trusted.

Shamir Secret Sharing, unlike all known asymmetric encryption
techniques, does not rely on computationally complexity of it's
security. Instead, it is fundamentally secure based on basic
information theory principles. Thus, unlike computationally secure
systems like RSA, Shamir Secret Sharing can not be broken regardless
of the amount of computational power one posses. Shamir Secret Sharing
functions on the basis of defining a polynomial of degree (K-1) over a
finite field with the Data D encoded as the first order-zero term. N
points are then selected from this polynomial and distributed to the
participants. Since K points (but no fewer) will uniquely identify the
original polynomial, and thus allow the deviation of D, K users must
combine their pieces in order to re-compute D.

Shamir Secret Sharing (and related systems) are useful in a wide range
of situations where one needs to distribute trust across multiple
entities. In particular, secret sharing techniques are leveraged in
some cryptographically-based access control systems like that
described in~\cite{Goyal2006}. Such systems will be discussed further
in Section~\ref{sec:ac}.

\subsection{State of the Art}

Both symmetric and asymmetric encryption systems have a place in the
modern security landscape: symmetric systems for their performance and
resistance to cryptanalysis and asymmetric systems for their
avoidance of the key exchange problem.  Often symmetric and asymmetric
cryptography are used together, each system playing to its
strength. Symmetric systems are good at quickly and securely
encryption data, making them appropriate for the core of an encryption
system. Symmetric ciphers, however, suffer from a lack of natively
secure method for exchanging the required encryption key. This is
where asymmetric cryptography and related secure key exchange systems
come in handy. These systems provide the basis for securely exchanging
data over insecure channels and they can be thus used to bootstrap
symmetric encryption systems by facilitating the secure exchange of a
symmetric encryption key which can then be used to encrypt the
underlying data. Such systems are common in many modern protocols like
SSL, TLS, PGP, and SSH.

Modern symmetric encryption ciphers like AES (Rijndael), Twofish, or
Camellia are well-established, fast, and secure methods for encrypting
data. Symmetric encryption systems are the preferred means of
encrypting files, hard disks, and other large chunks of data due to
their speed and relative simplicity of implementation. They are also
useful for encrypting streams of data in communication protocols like
TLS/SSL or SSH. They tend to be well understood, and are generally
considered highly secure (at least the well vetted ones). The strength
of a given symmetric cryptography is directly related to the length of
the associated encryption key. Common key lengths generally considered
secure today include 128-bit keys, 256-bit keys, 384-bit keys, and
512-bit keys.

Modern asymmetric encryption systems include systems like RSA,
ElGamal, ECDH, ECDSA, or ECIES. These systems are all built atop
various one-way trapdoor functions. RSA is based on
prime-factorization functions, ElGamal is based on discrete logarithm
functions, and ECDH, ECDSA, and ECIES are all based on various
elliptic curve functions. It is often useful to mix cipher suites
relying on different trapdoor functions as a hedge against an
efficient solution to any of the underlying function families being
discovered, thus rendering the encryption using such a family
obsolete. Within a given family the security of an asymmetric cipher
is related to the length of the keys in a key pair: longer keys are
more secure. Standard key lengths for asymmetric keys depend of the
family of functions be used. In prime-factorization based system like
RSA or discreet-log systems like ElGamal, key lengths of 1024-bits,
2048-bits, and 4096-bits are all common (with 2048 considered to be
best practice for general data and 4096 considered to be best practice
for highly sensitive data). Elliptic curve based systems tend to use
shorter keys: recommended sizes range from 160-bits to 512-bits
(similar to symmetric key lengths).

\subsection{Future Extensions}

Quantum

Randomness

Key management

\section{Access Control}
\label{sec:ac}

%% Over the years, we have developed a range of authentication techniques
%% and protocols. The goal of any authentication system is to confirm the
%% validity of a fact. In many authentication systems, the fact they aim
%% to confirm is the positive association between a user's asserted
%% identity and the user's actual identity. In short: is an actor who she
%% claims to be? Authentication systems can also be used to verify the
%% association between an actor and an object (i.e. does a user process
%% access to a specific token or device), between an actor and a
%% capability (i.e. as in a CAPTCHA~\cite{captcha}), or between a variety
%% of more generalized facts and associations. Authentication and
%% authorization are often complementary systems. Authentication is used
%% to establish the identity of an actor. Authorization then leverages
%% this identification as the basis of granting or denying specific
%% rights to the actor.

%% Early computer authentication schemes often revolved around the use of
%% a single basic primitive: text-based passwords. To this day, passwords
%% are probably the most common authentication primitive. Passwords are a
%% form of shared secret. They operate on the premises that only a
%% specific actor and the system with which she wishes to interact will
%% be aware of the value of a unique text token. When the actor wishes to
%% prove her identity, she provides her password to the system, which
%% confirms that it matches the expected password. Often, instead of
%% comparing passwords directly, passwords are first hashed before being
%% stored. Hashing provides some measure of security against attackers
%% wishing to brute-force a leaked password list. Hashing operates by
%% dissociating the provided password value from the stored password
%% value via a one-way hash function. Passwords are often used in an
%% interactive manner, where an actor must provide her password at a live
%% prompt. But passwords can also be used in non-interactive (albeit
%% generally less secure) forms where the necessary password is simply
%% stored and automatically provided when required. Passwords, while
%% common, have a range of known limitations and issues. From reuse, to
%% guess-ability, passwords have a lot of
%% problems~\cite{goodin-passwords, goodin-bible, Mazurek2013,
%%   Singer2013}. None the less, they remain ubiquitous authentication
%% primitives to this day due to their ease of use and user familiarity.

%% In addition to passwords, common authentication primitives also
%% include asymmetric cryptography certificates, multi factor devices,
%% biometrics, and contextual information. Systems like
%% OpenSSH~\cite{openSSH} and OpenPGP~\cite{openpgp} support using
%% standard asymmetric cryptography certificates as the basis for
%% authentication. In such systems, an actor's public key is stored by
%% the server. The user must prove they have access to the corresponding
%% private key, often by decrypting a message encrypted with their public
%% key, in order to authenticate. Certificate based authentication
%% systems have the benefit of often being non-interactive; the user must
%% simply posses the necessary certificate to gain access, no interactive
%% password prompt required. They also tend to be far more resistant to
%% brute-force attacks given the superior entropy of long,
%% randomly-generated certificates over short, human-generated passwords

%% Multi factor authentication systems are rising in popularity as a
%% mitigation tactic for the risks of password based authentication
%% systems. Such systems force the user to prove they have access to an
%% object (often a cell phone~\cite{google-auth} or USB
%% dongle~\cite{yubikey}) in addition to prompting the user for their
%% password or related primitive. Where as a password is ``something you
%% know'', a multi factor device is ``something you have'', the
%% combination of which make up the multiple factors in ``multi-factor''
%% authentication.

%% Biometric authentication systems have also become more common. Many
%% modern laptops and cell phone include fingerprint readers, and more
%% exotic devices like retina or palm scanners are not uncommon in
%% high-security installations. Systems have even been proposed that rely
%% on a user's unique keystroke patterns to identify
%% her~\cite{Peacock2004}. There are also a variety of contextual
%% authentication systems, that aim to authenticate the user on the basis
%% of various environmental data available when the user wishes to
%% authenticate (i.e. IP address, time of day,
%% etc)~\cite{Hulsebosch2005}. Such systems often provide a secondary
%% authentication mechanism beyond a primary mechanism like a password or
%% certificate.

%% Moving beyond basic authentication primitives, there are also a range
%% of existing authentication protocols and
%% standards. Kerberos~\cite{Kohl1994, Neuman1994} was an early and
%% widely deployed authentication system. It aims to provide secure
%% authentication over untrusted networks, as well as to allow
%% token-based single-sign-on access across multiple sites and
%% services. Kerberos is still used widely today as part of the Microsoft
%% suite of operating systems and in a number Linux and Unix
%% environments. Similarly, SAML (Security Assertion Markup
%% Language)~\cite{saml}, SASL (Simple Authentication and Security
%% Layer)~\cite{sasl} are standardized formats for exchanging
%% authentication and authorization data. SAML is the basis of
%% authentication systems like Shibboleth~\cite{shibboleth, Leandro2012}
%% whose aim is to create a standardized federated authentication system
%% for use across the Internet. Systems like OAuth~\cite{oauth},
%% OpenID~\cite{openid}, or Persona~\cite{persona} operate under a
%% similar principle, allowing users to designate a federated Cloud-based
%% identity providers who can be used to authenticate the user to a range
%% of disparate web services.

%% PAM~\cite{linux-pam, openpam} is a framework for integrating a variety
%% of authentication primitives and systems in an application. PAM is
%% used by Linux and a variety of other POSIX operating systems as the
%% basis for a flexible user login authentication system. PAM exposes a
%% standardized API for integrating various authentication technologies
%% into the a generalized authentication framework.

%% Custos aims to be flexible enough to incorporate a range of existing
%% authentication primitives and systems based on the user's
%% requirements. Custos also incorporates ideas from PAM related to the
%% pluggability of authentication modules. The details of these points
%% are discusses in subsequent chapters.

\section{Storage Security}
\label{sec:fs}

%% Early storage and file system technologies often simply neglected
%% security, lacking robust encryption and access control primitives.
%% Fortunately, today there are a variety of secure storage systems
%% available. Some of them are full stack systems that bundle security,
%% distribution, and sharing in a single system. Others are layered
%% systems, designed to add security atop existing lower level file
%% storage technologies. All of them have limitations that Custos strives
%% to overcome.

%% Many modern storage systems include cryptographic security as part of
%% their design. Such full stack systems bundle cryptography, distributed
%% usage, data storage, and other features into a single
%% package. Traditional network storage systems like
%% NFS~\cite{Sandberg1985} or AFS~\cite{Howard1988} provide support for
%% encrypting data as it travels over the network, but lack support for
%% encrypting data at rest, requiring users to fully trust the system on
%% which their data is stored or cached. Systems like
%% RFS~\cite{Dong2011}, Keypad~\cite{Geambasu2011}, or
%% CryptoCache~\cite{Jensen2000} are optimized for modern mobile device
%% usage, and include features like encryption at rest, auditing, and
%% multi-device support. Unfortunately, these systems lack support for
%% multi-user sharing. Systems like OceanStore~\cite{Kubiatowicz2000} or
%% Tahoe~\cite{Wilcox-O'Hearn2008} deal with securing data atop untrusted
%% infrastructure, and include primitives for securely sharing and
%% distributing files amongst users. These systems, however, lack support
%% for the kinds of out-of-band (e.g. emailing files, transferring files
%% on thumb drives, etc) sharing and syncing that are so common and
%% natural today. In general, full stack systems are only useful if you
%% are willing and able to utilize them as the entirety of your storage
%% stack, and are not easily extended or combined with other
%% technologies.

%% Other modern secure storage systems follow in the Unix tradition of
%% layered file systems, where each layer provides only a single function
%% (e.g. redundancy, encryption, storage, etc). Systems like
%% LUKS~\cite{luks} or eCryptfs~\cite{eCryptfs, Halcrow} are popular,
%% widely deployed, layered encryption systems. They are capable of
%% operating atop a variety of underlying file systems and are thus well
%% suited for use on personal computers. Most of these systems, however,
%% are not well suited for supporting secure multi-device syncing or
%% secure multi-user sharing.

%% All of the above systems, however, suffer from the traditional
%% entanglement of key management and the underlying encryption. As we
%% stated in the previous section, conflating these two items is to
%% conflate policy and mechanism, a well known sin in usable and
%% maintainable systems design~\cite{Wulf1974}. The bundling of key
%% storage with the underlying encryption leads to a lack of flexible key
%% management and access control capabilities. I am not the first to
%% recognize this barrier. The SFS~\cite{Mazieres1999} file system was
%% designed to separate key management from file storage, allowing for
%% more flexible key management in the process. Likewise,
%% Plutus~\cite{Kallahalla2003} strives toward separating key storage and
%% access control from the underlying encryption. But both SFS and Plutus
%% fail to fully define a standardized, generic, and flexible external
%% system for storing and managing keys, making a true ``Key Storage as a
%% Service'' architecture impossible to realize.

\section{Managing Security}
\label{sec:mgmt}

\section{Conclusion}
\label{sec:conclusion}

% Bibliography
\bibliographystyle{acm}
\bibliography{prelim}

\end{document}

%%  LocalWords:  BCE Diffe Hellman's Diffie Custos AFS CryptoCache
%%  LocalWords:  OceanStore SFS Plutus Adi Shamir ElGamal ECDH ECDSA
%%  LocalWords:  ECIES
