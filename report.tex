%\documentclass{acm_proc_article-sp}
\documentclass{sig-alternate}

\usepackage{url}
\usepackage{float}
\usepackage{caption}
\usepackage{subcaption}
\usepackage{hyperref}

\urlstyle{same}

\hypersetup{
    colorlinks,
    citecolor=black,
    filecolor=black,
    linkcolor=black,
    urlcolor=black
}

\newenvironment{packed_enum}{
\begin{enumerate}
  \setlength{\itemsep}{1pt}
  \setlength{\parskip}{0pt}
  \setlength{\parsep}{0pt}
}{\end{enumerate}}

\newenvironment{packed_item}{
\begin{itemize}
  \setlength{\itemsep}{1pt}
  \setlength{\parskip}{0pt}
  \setlength{\parsep}{0pt}
}{\end{itemize}}

\newenvironment{packed_desc}{
\begin{description}
  \setlength{\itemsep}{1pt}
  \setlength{\parskip}{0pt}
  \setlength{\parsep}{0pt}
}{\end{description}}

% --- Metadata ---
\permission{
  This work is licensed under the Creative Commons Attribution 4.0
  International License. To view a copy of this license, visit
  \url{http://creativecommons.org/licenses/by/4.0/}.
}
\conferenceinfo{CU CS Systems Prelim,}{
  Spring 2014. \\
  \crnotice{Copyright held by author(s).}
}
\copyrightetc{
  University of Colorado, Boulder \\
  \the\acmcopyr
}
\crdata{03/2014}
% --- End of Metadata ---

\begin{document}

\title{System and Data Security - A CS Systems Prelim}

\numberofauthors{1}
\author{
  \alignauthor
  Andy Sayler\\
  \affaddr{University of Colorado}
  \affaddr{Boulder, Colorado}
  \email{andy.sayler@colorado.edu}
}

\maketitle

\begin{abstract}
Security is a core component of modern computer systems. From
protecting our data to securing our communications, security across
the computing spectrum is fundamental to the manner in which we
leverage and trust computers. But the security of modern computing
systems has not come easily: it has been learned and improved slowly
over many years, sometimes at the cost of painful
lessons. Furthermore, the modern state of the art in computing
security still lea vs much to be desired. In this paper, I explore the
development of the current state of the art in computer security
focusing on four core components: cryptography, access control, file
system security, and security management. Computer security is an
inherently large topic, but these core topics provide a reasonable
basis for the modern state of computer security. In particular, I seek
to answer the question: ``How can we secure our systems and data in a
robust, comprehensive, and easy-to-use manner?''. This question in
examined from a historical perspective as well as the perspective of a
modern user with modern use cases. This paper builds on the background
work completed in my Master's Thesis~\cite{custos-masters}, further
extending my analysis of the current state of the art and
hypothesizing on future extensions to this state.
\end{abstract}

\section{Introduction}
\label{sec:intro}

We have reached the age of ubiquitous computing. There is not a facet
of our lives that is not heavily integrated with the vast computing
networks we have built and continues to expand. From our cell phones
we use to communicate to the web sites on which we mange our life's
savings to the hard drives in our laptops filled with our photos and
personal documents, computers are not only everywhere, but often at
the heart of our most intimate interactions. As such, the security of
our computing infrastructure is a foremost concern in modern computing
system design. But what is the state of the art in computing security?
And how have we arrived at this state? These are question I address in
this paper through the exploration and analysis of ten significant
publications in the field. In particular, I break my analysis of the
state of the art in computer security into four related topics:
cryptography, access control, storage security, and managing
security. These topics provide the basis of the bulk of the modern
state of the art of computer security.

On the topic of cryptography (Section \ref{sec:crypto}), I present the
basics or modern cryptographic systems~\cite{Diffie1976}, extensions
to these systems to accommodate the diversification of
trust~\cite{Shamir1979}, and the manner in which these core concepts
can be leverage in access control
applications~\cite{Bethencourt2007}. On the topic of access control
(Section \ref{sec:ac}), I present the basics of modern access control
models~\cite{Sandhu1996}, the ways in which these models can
incorporate cryptography to avoid the need for a trusted compute
base~\cite{Bethencourt2007}, and the manner in which various access
control schemes have been applied to modern file
systems~\cite{Miltchev2008}. On the topic of storage security (Section
\ref{sec:fs}), I present an effort to support file system distribution
with minimal trust~\cite{Mazieres1999}, an overview of the security
mechanism employed by a range of modern file system~\cite{Kher2005},
and the manners in which modern file system implement access
control~\cite{Miltchev2008}. Finally, on the topic of management
(Section \ref{sec:mgmt}), I present an early effort to standardize the
basic system security primitives~\cite{Samar1996}, techniques for
making security more robust and simpler for the end user to
leverage~\cite{Cox2002}, and modern efforts to unify security
primitives across multiple administrative
domains~\cite{Morgan2004}. These ten papers are by no means the
complete body of prior art, but they do elucidate the core concepts
relevant to the question at the core of this exam.

In addition to exploring the historic contributions of the work
mentioned above and the current state of the art they represent, I
also suggest possible future expansions of this state. At it's core,
this involves looking at the existing answers to the question
question: ``How can we secure our systems and data in a robust,
comprehensive, and easy-to-use manner?'' as well as proposing
potential new answers. This hypothesizing is addressed with respect to
the four core topics mentioned above within each of the relevant
sections.

\section{Cryptography}
\label{sec:crypto}

\section{Access Control}
\label{sec:ac}

\section{Storage Security}
\label{sec:fs}

\section{Managing Security}
\label{sec:mgmt}

\section{Conclusion}
\label{sec:conclusion}

% Bibliography
\bibliographystyle{acm}
\bibliography{proposal}

\end{document}
