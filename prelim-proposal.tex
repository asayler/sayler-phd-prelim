%\documentclass{acm_proc_article-sp}
\documentclass{sig-alternate}

\usepackage{url}
\usepackage{float}
\usepackage{caption}
\usepackage{subcaption}
\usepackage{hyperref}

\urlstyle{same}

\hypersetup{
    colorlinks,
    citecolor=black,
    filecolor=black,
    linkcolor=black,
    urlcolor=black
}

\newenvironment{packed_enum}{
\begin{enumerate}
  \setlength{\itemsep}{1pt}
  \setlength{\parskip}{0pt}
  \setlength{\parsep}{0pt}
}{\end{enumerate}}

\newenvironment{packed_item}{
\begin{itemize}
  \setlength{\itemsep}{1pt}
  \setlength{\parskip}{0pt}
  \setlength{\parsep}{0pt}
}{\end{itemize}}

\newenvironment{packed_desc}{
\begin{description}
  \setlength{\itemsep}{1pt}
  \setlength{\parskip}{0pt}
  \setlength{\parsep}{0pt}
}{\end{description}}

% --- Metadata ---
\permission{
  This work is licensed under the Creative Commons Attribution 4.0
  International License. To view a copy of this license, visit
  \url{http://creativecommons.org/licenses/by/4.0/}.
}
\conferenceinfo{CU CS Systems Prelim,}{
  Spring 2014. \\
  \crnotice{Copyright held by author(s).}
}
\copyrightetc{
  University of Colorado, Boulder \\
  \the\acmcopyr
}
\crdata{03/2014}
% --- End of Metadata ---

\begin{document}

\title{System and Data Security - A CS Systems Prelim}
\subtitle{PROPOSAL}

\numberofauthors{1}
\author{
  \alignauthor
  Andy Sayler\\
  \affaddr{University of Colorado}
  \affaddr{Boulder, Colorado}
  \email{andy.sayler@colorado.edu}
}

\maketitle

\section{Proposal}

I propose that my CS System PhD Prelim exam focus on system and data
security. My exam will build on the background work completed in my
Master's Thesis~\cite{custos-masters}, further extending my analysis
of the current state of the art and hypothesizing on future extensions
to this state. This prelim will be overseen by Prof. Dirk Grunwald,
with the help of committee members Prof. Eric Keller and Prof. John
Black.

This exam will approach the topic of system and data security from
four core areas: cryptography, access control, file systems, and
management. System and data security is an inherently large topic:
narrowing the focus to these core topics will help to keep the exam
size manageable. In particular, this exam seeks answers to the
question: ``How can we secure our systems and data in a robust,
comprehensive, and easy-to-use manner?''. This question in examined
from a historical perspective as well as the perspective of a modern
user with modern use cases.

This exam will focus on the work presented in ten papers
representative of the prior art. On the topic of cryptography, I will
present the basics or modern cryptographic systems~\cite{Diffie1976},
extensions to these systems to accommodate the diversification of
trust~\cite{Shamir1979}, and the manner in which these core concepts
can be leverage in access control
applications~\cite{Bethencourt2007}. On the topic of access control, I
will present the basics of modern access control
models~\cite{Sandhu1996}, the ways in which these models can
incorporate cryptography to avoid the need for a trusted compute
base~\cite{Bethencourt2007}, and the manner in which various access
control schemes have been applied to modern file
systems~\cite{Miltchev2008}. On the topic of file systems, I will
present an effort to support file system distribution with minimal
trust~\cite{Mazieres1999}, an overview of the security mechanism
employed by a range of modern file system~\cite{Kher2005}, and the
manners in which modern file system implement access
control~\cite{Miltchev2008}. Finally, on the topic of management, I
will present an early effort to standardize the basic system security
primitives~\cite{Samar1996}, techniques for making security more
robust and simpler for the end user to leverage~\cite{Cox2002}, and
modern efforts to unify security primitives across multiple
administrative domains~\cite{Morgan2004}. These ten papers are by no
means the complete body of prior art, but they do elucidate the core
concepts relevant to the question at the core of this exam.

% Bibliography
\bibliographystyle{acm}
\bibliography{prelim}

\end{document}
